\documentclass[]{article}

\title{Jane Street May 2022 - Robot Updated Swimming Trials}
\author{Miguel Barbosa Pereira}

\usepackage{mathtools}
\usepackage{amssymb}
\usepackage{amsthm}
\usepackage{IEEEtrantools}
\usepackage{geometry}
\usepackage{parskip}
\usepackage{xcolor}
\usepackage{hyperref}               % Include hyperlinks in the PDF
\usepackage{bookmark}               % Upgrades hyperref
\usepackage[capitalise]{cleveref}   % Smart referencing
\usepackage[all]{hypcap}            % Make hyperlinks to floats point to the top of the float instead of the caption

\newlength{\alphabet}
\settowidth{\alphabet}{\normalfont abcdefghijklmnopqrstuvwxyz}
\geometry{textwidth=3\alphabet,textheight=4.5\alphabet,hcentering}
\allowdisplaybreaks
\hypersetup{
    bookmarksnumbered=true,
    colorlinks=true,
    linkcolor=blue,
    citecolor=blue,
    urlcolor=blue
}
% --------------------
% Theorem declarations
% --------------------
\theoremstyle{plain}      % Theorem-like environments - theorem
\newtheorem{theorem}     {Theorem}
\newtheorem{proposition} [theorem] {Proposition}
\newtheorem{lemma}       [theorem] {Lemma}
\newtheorem{corollary}   [theorem] {Corollary}
\newtheorem{conjecture}  [theorem] {Conjecture}

\theoremstyle{definition} % Theorem-like environments - definition
\newtheorem{exercise}   [theorem] {Exercise}
\newtheorem{definition} [theorem] {Definition}
\newtheorem{example}    [theorem] {Example}
\newtheorem{remark}     [theorem] {Remark}
\newtheorem{assumption} [theorem] {Assumption}
\newtheorem{axiom}      [theorem] {Axiom}

\begin{document}

\maketitle

\textbf{Problem:} Back in October, we were tasked with the puzzle of finding the smallest number of robots we could invite to a Robot Swimming Trial in order to change the optimal strategy of participants away from discrete. Click here to see the original puzzle explanation and example, in the next paragraph we will go over a quick refresher on the setup.

The tournament directors choose a positive integer $n$, and then $3n$ robots are invited to compete in the trials, which are $n$ races between all $3n$ robots. The robots commit to using a schedule of their identical fuel amounts to the $n$ races, and on a given race whatever robot burns the most fuel wins (ties for most fuel spent on a given race are split uniformly randomly among the tying robots). All robots swim in all races according to their schedules, and then $n$ distinct winners are determined, one from each race, by successively selecting the robot from the remaining races that spent the most fuel and finished ahead of all other robots that haven't yet won a race. Robots all compete (i.e. choose their fuel allotment distributions to maximize the probability) to be selected for the finals by this method. The discrete strategy is the one in which a robot chooses a race uniformly randomly and assigns all of its fuel to that race, and zero fuel to the other races.

We found (spoiler alert!) that inviting 24 robots to compete in 8 trial races (i.e. $n$=8) created a tournament in which the discrete strategy was no longer optimal. The tournament organizers adopted this tournament size, and viewers were rewarded with a richer set of strategies and more diverse race results\footnote{Also, many fewer extremely slow races in which all robots devoted zero fuel, and whichever robot's random motion took it across the finish line first was declared the winner.}. As expected, the competitors switched from everyone always using the discrete strategy to more subtle and continuous allotments of fuel. The metagame evolved and eventually settled into a Nash equilibrium in which a given competitor chooses the discrete strategy with a certain probability $p$, and otherwise elects for an allotment that distributes nonzero fuel to at least two races.

Find $p$, the probability a robot using the Nash equilibrium strategy when competing at a Robot Updated Swimming Trial with $n=8$ devotes all of its fuel to a single race, to 6 significant digits.

\begin{definition}[parameters of the problem]
    \begin{IEEEeqnarray*}{rCls+x*}
        n & \coloneqq & 8 = \text{number of races}, \\
        m & \coloneqq & 3 = (\text{number of robots}) / (\text{number of races}), \\
        p & \coloneqq & \text{probability that a robot uses a strategy which is discrete} \in (0,1).
    \end{IEEEeqnarray*}
\end{definition}

\begin{definition}[strategies]
    \begin{itemize}
        \item[]
        \item A \textbf{pure strategy} is a tuple $(s_1, \ldots, s_n) \in [0,1]^n$ such that $s_1 + \cdots + s_n = 1$. Let $X$ be the set of pure strategies.
        \item A pure strategy is \textbf{discrete} if there exists $i = 1, \ldots, n$ such that $s_i = 1$. A pure strategy is \textbf{continuous} if it is not discrete. Denote by $D \subset X$ the set of discrete pure strategies, $C \subset X$ the set of continuous pure strategies, and by $D_i \in X$ the discrete pure strategy $(0,\ldots,0,i,0,\ldots,0)$.
        \item A \textbf{mixed strategy} is a probability measure $\mu$ on $X$.
    \end{itemize}
\end{definition}

Now let $\mu_1, \ldots, \mu_{mn}$ be the strategies of the $1, \ldots, mn$ robots.

\begin{definition}[probability space]
    Consider the probability space $(\Omega_0, P_0) \coloneqq (X^{mn}, \mu_1 \times \cdots \times \mu_{mn})$. Let $(\Omega, P)$ be the probability space obtained from $(\Omega_0, P_0)$ by replacing a point which corresponds to a tie between the robots by many points, one for each possible outcome of the tie.
\end{definition}

\begin{definition}[events]
    Let $R \in \{R_1,\ldots,R_{mn}\}$ be a robot and $S \subset X$ be a set of pure strategies.
    \begin{IEEEeqnarray*}{rCls+x*}
        W(R)   & \coloneqq & \{\omega \in \Omega \mid \text{at $\omega$, robot $R$ is one of the winners}\}, \\
        W_i(R) & \coloneqq & \{\omega \in \Omega \mid \text{at $\omega$, robot $R$ wins before race number $i$}\}, \\
        S(R)   & \coloneqq & \{\omega \in \Omega \mid \text{at $\omega$, robot $R$ is using a pure strategy in $S$}\}.
    \end{IEEEeqnarray*}
\end{definition}

\begin{axiom}
    \begin{itemize}
        \item[]
        \item The strategies $\mu_1, \ldots, \mu_{mn}$ correspond to a Nash equilibrium.
        \item For any robot $R$, we have $P(D(R)) = p$ and $P(C(R)) = 1 - p$.
    \end{itemize}
\end{axiom}

\begin{lemma}
    \label{lem:nash eq}
    Let $R$ be a robot and $S \subset X$ be a set of pure strategies such that $\mu_R(S) > 0$. Then $P(W(R)) = P(W(R) \mid S(R))$.
\end{lemma}
\begin{proof}
    Assume by contradiction that $P(W(R)) \neq P(W(R) \mid S(R))$.

    In the case where $P(W(R)) > P(W(R) \mid S(R))$, robot $R$ would be better off by never using the strategies in $S$, which contradicts the fact that we are in a Nash equilibrium.
    
    In the case where $P(W(R)) < P(W(R) \mid S(R))$, robot $R$ would be better off by always using the strategies in $S$, which contradicts the fact that we are in a Nash equilibrium.
\end{proof}

\begin{corollary}
    \label{cor:nash eq}
    If $R$ is a robot and $i \in \{1,\ldots,n\}$, then
    \begin{IEEEeqnarray*}{rCls+x*}
        P(W(R))
        & = & P(W(R) \mid D_i(R)) \\
        & = & P(W(R) \mid D(R)) \\
        & = & P(W(R) \mid C(R)).
    \end{IEEEeqnarray*}
\end{corollary}

\begin{lemma}
    \label{lem:how many robots}
    Let $E \subset \Omega$ be any event, $S_i \subset X$ be a set of pure strategies for $R_i$ (for each $i = 1,\ldots,mn$) and $F_i$ be the event
    \begin{IEEEeqnarray*}{c+x*}
        F_i \coloneqq \{\omega \in \Omega \mid \text{at $\omega$, $i = \# \{ j \in \{2, \ldots, mn\} \mid R_j \text{ uses a strategy in } S_j \}$}\},
    \end{IEEEeqnarray*}
    for each $i = 0, \ldots, mn-1$. Then $P(E \mid S_1(R_1)) = \sum_{i=0}^{mn-1} P(F_i) P(E \mid S_1(R_1) \cap F_i)$ (and analogously for robots other than $R_1$).
\end{lemma}
\begin{proof}
    \begin{IEEEeqnarray*}{rCls+x*}
        \IEEEeqnarraymulticol{3}{l}{P(E \mid S_1(R_1))}\\ \quad
        & = & \frac{1}{P(S_1(R_1))} P(W(R_1) \cap S_1(R_1))                                                            & \quad [\text{conditional probability}]\\
        & = & \frac{1}{P(S_1(R_1))} P\left(W(R_1) \cap S_1(R_1) \cap \bigcup_{i=0}^{mn-1} F_i \right)                  & \quad [\text{$\Omega = \bigcup_{i=0}^{mn-1} F_i$}] \\
        & = & \frac{1}{P(S_1(R_1))} P\left(\bigcup_{i=0}^{mn-1} W(R_1) \cap S_1(R_1) \cap F_i \right)                  & \quad [\text{De Morgan}]\\
        & = & \frac{1}{P(S_1(R_1))} \sum_{i=0}^{mn-1} P\left( W(R_1) \cap S_1(R_1) \cap F_i \right)                    & \quad [\text{disjoint events}]\\
        & = & \sum_{i=0}^{mn-1} \frac{P(S_1(R_1) \cap F_i)}{P(S_1(R_1))} P\left( W(R_1) \mid S_1(R_1) \cap F_i \right) & \quad [\text{conditional probability}] \\
        & = & \sum_{i=0}^{mn-1} P(F_i) P\left( W(R_1) \mid S_1(R_1) \cap F_i \right)                                   & \quad [\text{independent events}]. & \qedhere
    \end{IEEEeqnarray*}
\end{proof}

\begin{lemma}
    \label{lem:win race 1}
    If $R$ is a robot, then $P(W_1(R) \mid D_1(R)) = \frac{1}{mp} \left(1 - \left(1 - \frac{p}{n}\right)^{mn}\right)$ (and analogously for other races other than the first).
\end{lemma}
\begin{proof}
    Let $S_i = D_1$ and define $F_i$ as in \cref{lem:how many robots}.
    \begin{IEEEeqnarray*}{rCls+x*}
        \IEEEeqnarraymulticol{3}{l}{P(W_1(R_1) \mid D_1(R_1))}\\ \quad
        & = & \sum_{i=0}^{mn-1} P(F_i) P(W_1(R_1) \mid S_1(R_1) \cap F_i) \\
        & = & \sum_{i=0}^{mn-1} b(i, mn-1, p/n) \frac{1}{1+i}                                                                           & \quad \\
        & = & \sum_{i=0}^{mn-1} \binom{mn-1}{i} \left(\frac{p}{n}\right)^i \left(1 - \frac{p}{n}\right)^{mn-1-i} \frac{1}{1+i}          & \quad [\text{formula for $b$}] \\
        & = & \sum_{j=1}^{mn} \binom{mn-1}{j-1} \left(\frac{p}{n}\right)^{j-1} \left(1 - \frac{p}{n}\right)^{mn-j} \frac{1}{j}          & \quad [\text{$j = i + i$}]\\
        & = & \frac{n}{p} \frac{1}{mn} \sum_{j=1}^{mn} \binom{mn}{j} \left(\frac{p}{n}\right)^{j} \left(1 - \frac{p}{n}\right)^{mn-j} \\
        & = & \frac{1}{mp} \sum_{j=1}^{mn} b(j,mn,p/n)                                                                                  & \quad [\text{formula for $b$}]\\
        & = & \frac{1}{mp} \left(\sum_{j=0}^{mn} b(j,mn,p/n) - b(0,mn,p/n) \right) \\
        & = & \frac{1}{mp} \left(1 - \left(1-\frac{p}{n}\right)^{mn} \right).                                                           &                                                    & \qedhere
    \end{IEEEeqnarray*}
\end{proof}

\begin{lemma}
    \label{lem:prob winning bound}
    If $R$ is a robot, then
    \begin{IEEEeqnarray*}{c+x*}
        P(W(R)) \geq \frac{1}{mp} \left(1 - \left(1 - \frac{p}{n}\right)^{mn}\right).
    \end{IEEEeqnarray*}
\end{lemma}
\begin{proof}
    \begin{IEEEeqnarray*}{rCls+x*}
        P(W(R))
        & =    & P(W(R) \mid D_1(R))                                             & \quad [\text{by \cref{cor:nash eq}}] \\
        & \geq & P(W_1(R) \mid D_1(R))                                           & \quad \\
        & =    & \frac{1}{mp} \left(1 - \left(1 - \frac{p}{n}\right)^{mn}\right) & \quad [\text{by \cref{lem:win race 1}}]. & \qedhere
    \end{IEEEeqnarray*}
\end{proof}

\begin{lemma}
    \label{lem:p geq 0.9}
    $p \geq 0.95223967$.
\end{lemma}
\begin{proof}
    There exists a robot $R$ such that $P(W(R)) \leq 1/m$ (otherwise, every robot would have probability of winning $> 1/m$, which is not possible). Then,
    \begin{IEEEeqnarray*}{l+x*}
        \frac{1}{m} \geq P(W(R)) \geq \frac{1}{mp} \left(1 - \left(1 - \frac{p}{n}\right)^{mn}\right) \\
        \quad \Longrightarrow \left(1 - \frac{p}{n}\right)^{mn} + p - 1 \geq 0 \\
        \quad \Longrightarrow p \geq 0.95223967. & \qedhere
    \end{IEEEeqnarray*}
\end{proof}

\end{document}